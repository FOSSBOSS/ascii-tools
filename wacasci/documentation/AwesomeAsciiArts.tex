\documentclass[12pt]{article}

\usepackage[T1]{fontenc}
\usepackage{lmodern}
\usepackage{hyperref}
\usepackage{geometry}
\usepackage{enumitem}
\usepackage{array}
\usepackage{graphicx}
\usepackage{titlesec}

\geometry{margin=1in}
\setlength{\parskip}{0.8em}
\setlength{\parindent}{0pt}

\title{\textbf{Awesome ASCII Arts}\\[0.3em]
	\large A Lightweight Qt-based ASCII Drawing Tool}
\author{Sensored Hacker}
\date{}

\begin{document}
	
	\maketitle
	
	\begin{abstract}
	Q: When is a text editor \emph{not} a text editor?  \\
	A:When it becomes an awesome ASCII art drawing application.

	\end{abstract}
	
	\section*{Why?}
	This tool exists for one reason: \textbf{fun}.  
	The application provides a simple, enjoyable way to draw ASCII art using the mouse or a tablet, trace images, and export clean text files. It is open-source, and requires no special hardware.
	\section*{Features}
	\begin{itemize}[leftmargin=2em]
		\item Mouse or tablet drawing
		\item Save drawings as plain text files
		\item A customizable panel of brushes
		\item Transparent window for tracing on-screen images
		\item On-screen color options for visibility
		\item Selectable character brushes 
		\item Open-source Python/Qt code—modify freely
	\end{itemize}
	
	\section*{Default Key Bindings}
	\begin{center}
		\begin{tabular}{>{\bfseries}l l}
			T & Brush 0 (``\texttt{...oooOOO000OOOooo...}'') \\
			Y & Brush 1 (``\texttt{...,,,---'''}'') \\
			U & Brush 2 (``\texttt{[[[]]]|||///\textbackslash\textbackslash}'' ) \\
			I & Brush 3 (``\texttt{HACKTHEMATRIX}'') \\
			J & Brush 4 (``\texttt{]]} '') \\
			K & Brush 5 (``\texttt{boob} '') \\
			L & Brush 6 (``\texttt{??} '') \\
			M & Brush 7 (``\texttt{---} '') \\
			N & Brush 9 (``\texttt{...} '') \\
			E & Eraser \\
			R & Reset canvas \\
			P & Populate entire canvas with text \\
			B & Toggle background/transparency \\
			Ctrl+S & Save ASCII file \\
			H & Open help window \\
			1--9 & Change on-screen text color \\
			0 & White text color \\
			Mouse & Draw characters \\
		\end{tabular}
	\end{center}
	
	\begin{figure}[h!]
		\centering
		\includegraphics[width=0.6\textwidth]{help.png}
		\caption{Help window overlay with transparent window in the background}
	\end{figure}
	
	\section*{Default Behavior}
	Launching the application with no arguments opens an 80×80 character transparent window.  
	Drawing begins immediately with the default brush. You may begin drawing with your mouse, tablet, arrow keys, WASD keys or all of the above.
	
	\section*{Command-Line Arguments}
	\begin{itemize}
		\item \texttt{python QAAA.py WIDTH HEIGHT}  
		Opens a window of the specified character dimensions.
	\end{itemize}
	
	Example:
	\begin{quote}
		\texttt{python QAAA.py 120 60}
	\end{quote}
	
	\section*{Future Development}
	Planned enhancements:
	\begin{itemize}
		\item A more comprehensive control panel.
		\item \textbf{stamps:} multi-character matrices. 
		\item In-place text editing

	\end{itemize}
	
	\section*{Compatibility}
	Although developed and tested on Linux, the application should run on any platform supported by Python 3 and Qt5, including:
	\begin{itemize}
		\item GNU/Linux
		\item macOS
		\item Windows
		\item BSD variants
	\end{itemize}
	
	\section*{System Requirements}
	\begin{itemize}
		\item Python 3.x
		\item PyQt5 
	\end{itemize}
	
	\section*{Active Keys:}
	
		\begin{figure}[h!]
		\centering
		\includegraphics[width=0.8\textwidth]{q.png}
		\caption{Current key bindings}
		%mostly to impress on me why we need a full feature control panel
	\end{figure}

    \section{Conclusion:}
    QAAA: Qt based Ascii Arts app or whatever its called is a text editor that wants to draw. It has a unique take on what priority features are. 
	%\section{Source Code}
	%You may include code listings here using the \texttt{listings} or \texttt{minted} packages.
	
	\vfill
	\begin{center}
		\textit{Have fun creating awesome ASCII art!}
	\end{center}
	
\end{document}